%\documentclass[a4paper,11pt]{article}
\documentclass[twocolumn]{IEEEtran}
\usepackage[pdftex]{graphicx}
\usepackage{amsmath}
\usepackage{amsthm}
\usepackage{cite}
\usepackage{array}
\usepackage{algorithm2e}
\usepackage{url}
\usepackage{xcolor}
\input{notation.inc}
\newcommand{\falta}[1]{{\color{red}FALTA:#1}}

\title{Binary Matrix Factorization via Dictionary Learning}\author{Ignacio Ram\'{i}rez}
%
%
%
\begin{document}
%
\maketitle
%
\begin{abstract}
Matrix factorization is a key tool in data analysis; its applications include recommender systems, correlation analysis, signal processing, among others. Binary matrices are a particular case which has received significant attention for over thirty years, especially within the field of data mining. Dictionary learning refers to a family of methods for learning overcomplete basis (also called frames) in order to efficiently encode samples of a given type; this area, now also about twenty years old, was mostly developed within the signal processing field.
In this work we propose a number of binary matrix factorization methods which are based on an adaptation of the dictionary learning paradigm to binary matrices.
The proposed algorithms focus on speed and scalability; they work with binary factors combined with bit-wise operations and a few auxiliary integer ones.
Another important issue in matrix factorization is the choice of rank for the factors; we address this model selection problem using the Minimum Description Length principle and propose a number of methods for efficiently choosing the best rank.
Our preliminary results show that the proposed method is effective at producing interpretable factorizations of various data types of interest. We also show the performance of the method on a simple binary denoising application, obtaining promising results.
\end{abstract}%
\begin{IEEEkeywords}
binary matrix factorization, binary dictionary learning, data mining
\end{IEEEkeywords}
%
\section{introduction}

We consider the problem of approximating a binary matrix $X \in \{0,1\}^{m{\times}n}$ as the product of two other binary matrices $U \in \{0,1\}^{m{\times}k}$ and $V \in \{0,1\}^{n{\times}p}$ plus a third \emph{residual} matrix $E$,
\begin{equation}
X = UV\transp + E.
\label{eq:mf}
\end{equation}
In this work we will consider the $n$ \emph{columns} of $X$ as the data samples to be represented, each of dimension $m$. (Note that this convention, typical in signal processing, is opposite to most works in statistics and computer science, where the \emph{rows} of $X$ are the samples.)
%The problem \refeq{eq:mf} has been addressed using various methods from different fields for over a hundred years. A famous example is Principal Component Analysis (PCA), pioneered by Pearson~\cite{pca}. The PCA method is based on the Singular Value Decomposition (SVD) of a matrix, $X=U{\Sigma}V\transp$ where $U$ and $V$ are matrices of $p=\min\{m,k\}$  orthogonal columns each and $\Sigma$ is a diagonal $p{\times}p$ matrix whose diagonal elements $\sigma_{11} \geq \sigma_{22} \geq \ldots \geq \sigma_{pp} \geq 0$, are the so-called \emph{singular values} of $X$. The PCA method decomposes $X$ by selecting the first $p' \leq p$ columns of $U$, $V$, and upper-left $p'{\times}p'$ submatrix of $\Sigma$ to form an approximation of $X$, $X'=U'\Sigma'(V')\transp$. It can be shown that $X'$ is the rank-$p'$ matrix that best approximates $X$ in Frobenius norm.
%
%Dozens of variants, with different properties, have been developed since PCA was first proposed. Some methods target the robustness of PCA against perturbations in the entries of $X$. This idea was first proposed in~\cite{rpca} and is embodied today by many methods under what is called Robust or Generalized PCA; see~\cite{rpca-paper,rpca-book} for more on the subject. Other methods impose additional restrictions on the factors; a common  one is to enforce $U$ and $V$ in \refeq{eq:mf} to be non-negative, giving rise to the so called Nonnegative Matrix Factorization (NMF) family of methods~\cite{nmf-review}. The restriction of non-negativity is often imposed to improve interpretability  or because the factors are expected to be non-negative by the nature of the problem to be solved (e.g., probabilities). For similar reasons, in different contexts, some matrices cannot be reasonably assumed to be real numbers. If the matrix entries should be integer (or naturals), the usual approach is to treat them as real (or non-negative) numbers anyway and then truncate the result. In other situations, only a relative order between the values can be established. A  more complicated setting is when the entries are just categories or labels (such as Football Teams) for which there is no reasonable relationship  but ``equal'' or ``different''. We call such matrices \emph{categorical}. 

% bmf07 - Zhang, NML proyectado en 0,1. Garantias de recup. bajo hipotesis
% bmf06 -Message Passing for BMF. Además de original, tiene FLOR de review en BMF
% bmf-mdl - ASSO (algoritmo interesante de BMF); MDL casi idéntico al mio
% bmf13 - semi-binario, resuelto con optimización tradicional, pesado, pero
%         con pruebas de solucion exacta en algunos casos.
% falta tiling!!
% bmf-dm - primer paper de Mittienen o algo asi de aplicacion de BMF a data mining
% monson95 - inconseguible, pero citado como 'review de enfoque combinatiorio a BMF
% ASSO: association rule es lo que yo puse como 'correlacion'!
%
The problem of Binary Matrix Factorization (BMF) problem has been treated extensively in the last three decades. It was first studied as a combinatorial problem (see~\cite{monson95} and references therein). It then received great attention from the data mining community starting with works such as~\cite{proximus,tiling} which fall into the category of \emph{tile matching} heuristics, where binary matrices are decomposed as superpositions of square tiles, and which are in fact also closely related to  the matrix factorization paradigm (note that a tile is nothing but the rank-one outer product of two binary vectors) and~\cite{asso}, which was the first to pose the binary data mining problem as a matrix factorization one.

A thorough survey of BMF methods is beyond the scope of this paper; we refer the reader to~\cite{bmf-mdl} which incidentally proposes a very similar MDL-based model selection strategy for selecting the best rank of the decomposition (more on this later). We will however mention some key works which represent either recent developments or different approaches to the problem.

\subsection{Binary Matrix Factorization}

We begin with the ASSO algorithm proposed in~\cite{asso}. The method in question constructs a binary correlation matrix $C$ by thresholding the correlation matrix $C=A{\transp}A$ with parameter $\tau$. It then produces a series of increasing rank approximations by adding a column taken from $C$ and a row which is chosen to maximize, together with that column, a given covering function (essentially, how many bits are set to zero in the residual). As in our case, this method can be efficiently implemented with bitwise and integer operations. It is also a popular and simple method with good performance. However, it does not scale well with the number of samples. In particular, for dictionary learning applications where $n \gg m$, the matrix $C$ to be computed can be extremely large, and the overall complexity of each step is $O(kn^2m)$, also very large for values of $n$ which in our case can be over a million.

The work~\cite{bmf07} is one of many examples which use nonlinear optimization over real spaces to solve BMF as a relaxed problem, that is, a Nonnegative Matrix Factorization formulation with values restricted to lie in the range $[0,1]$. The problem solved by~\cite{bmf07} is actually convex and can be solved exactly. Furthermore, if one assumes that $X$ is a noisy observation of a low-rank matrix $\hat{X}$ corrupted by a sparse noise matrix $E$, the solution obtained can be shown to coincide with the sought matrix $\hat{X}$. However, even with modern nonlinear optimization tools, the problem to be solved is computationally very demanding, requiring, as most such methods, repeated computations of the Singular Value Decomposition of a matrix of the same size as $X$.
% bmf13 - semi-binario, resuelto con optimización tradicional, pesado, pero
%         con pruebas de solucion exacta en algunos casos.
% proximus
The work~\cite{bmf13} also poses the BMF problem in terms of a relaxed one, but in this case only one of the factors is assumed binary, while the other is only assumed to be non-negative. Here too, under certain hypothesis regarding an unobserved matrix to be recovered, the result can be shown to coincide with the desired unobserved matrix.

The work~\cite{bmf06} stands out as an interesting alternative to BMF which formulates the decomposition of $X$ as a Bayesian denoising model with a particular prior on the unobserved \emph{clean} matrix $\hat{X}$ (again, $X=\hat{X}+E$) and uses a Message Passing algorithm to find the maximum a posteriori estimation of $\hat{X}$.

As an example of a tile-searching method we mention the Proximus method~\cite{proximus}, which approximates the first principal left and right binary components of the binary matrix $X$. The method can be extended to produce a hierarchical representation of the matrix with further rank-$1$ components, although these do not coincide with additional factors in a rank-$k$ factorization. Despite its simplicity, this is a very fast method compared to any of the above methods. Also, as we will see in the next subsection, finding the first principal component of a binary matrix is closely related to a crucial step in one of the main dictionary learning methods. Therefore, we will describe this method in detail later in this document.

\subsection{Dictionary learning}

The second pillar of our proposed solution is inspired by the \emph{dictionary learning} techniques whose origins date back to the 2000's~\cite{lewicki99,engan00,aharon06} within the field of signal processing

\subsection{Main contribution}

This work proposes a novel method for binary matrix factorization which combines ideas and tools from dictionary learning and data mining methods while relying entirely on binary algebras which are in turn implemented using the standard bit-wise operators available in any microprocessor, plus the vectorized operations available in most modern processors~\footnote{Of particular impact is the recent addition of the single-instruction \texttt{popcount} operation to Intel processors, which counts the number of ones in a processor register, thus makin summation and inner products with binary operands even more efficient.}. The result is an extremely fast and scalable method, specially when compared to its real-relaxation counterparts. 

Despite the restriction to binary operations, the convergence guarantees offered by our method are no worse than those of most matrix factorization methods (with the exception of PCA, which can be solved exactly) and, in particular, dictionary learning;  as most MF problems are non-convex, the solutions can be guaranteed to be locally optimal at best, and this carries out to our case as well. Furthermore, being of discrete nature, our method is guaranteed to converge in a finite number of steps. This is also a crucial advantage over generic MF methods in terms of speed and accuracy.

Finally, we show that the results obtained with our method are clearly interpretable in a variety of cases; we also show how to apply them to perform standard restoration tasks such as classification and correcting noisy or corrupt datum (denoising).

The rest of this document is organized as follows: Section~\ref{sec:dictionary-learning} is devoted to describing the basic ideas and methods in dictionary learning. Section~\ref{sec:proximus} describes the Proximus method, from which some key steps of our algorithm and its properties are derived. Section~\ref{sec:bdl} describes our proposed methods, their properties, discusses some interesting variants.
Section~\ref{sec:model-selection} deals with the problem of model selection (which in in this case amounts to choosing the rank of the factors) and proposes a solution based on the Minimum Description Length~\cite{mdl1,mdl2,mdl3} approach (which, as we mentioned, is very similar to that described in~\cite{bmf-mdl}).
 Section~\ref{sec:denoising} showcases how to apply the result of our method to the problem of binary signal  denoising.
We present and discuss the results obtained on those problems in Section~\ref{sec:results}, and provide concluding remarks in Section~\ref{sec:conclusion}.

\section{Dictionary Learning and sparsity}
\label{sec:dictionary-learning}
%
The aim of \emph{Dictionary Learning} methods is to obtain efficient representations of specific types of signals in terms of basis or \emph{frames} which are adapted to the particularities of the signal type at hand (see~\cite{dl-review} for a review). This is in contrast to the traditional approach where fixed transforms such as the Discrete Fourier Transform (DFT) or any of the wide family of Discrete Wavelet Transforms (DWT) are used to compactly represent the salient data of signals such as audio tracks or natural images. In this setting, the columns of $X$ are data samples, for example short audio segments or pixel values from patches of natural images. The dimension $m$ of the data samples is usually much smaller than the number of samples $n$. In the more general context of matrix factorizations, we say that these are extremely \emph{fat matrices}; this is an important aspect to consider in MF algorithms, as we will see later.
For sufficiently large $n$, we can tailor a \emph{dictionary} $D \in \reals{m{\times}p}$ such that each sample in $X$ can be compactly represented in terms of a few columns of $D$; by compact we mean that we can represent $X=DA+E$ with $\|E_i\| \ll \|X_i\|$ and a coefficients column $A_i\|$ which has a few non-zero values. Depending on the context, we will use two equivalent notations to denote the number of non-zeros in a vector $v$: one is the  $\ell_0$ pseudo-norm, $\|A_i\|_0$, and the other is the Hamming weight, $h(v)$; $h(v)=\|v\|_0=|\{i:A_i \neq 0\}|$, where $|G|$ denotes the number of elements in the set G.

As we can see from the previous discussion, contrary to the more general MF approach where the factors $U$ and $V$ have similar roles, those of $D$ and $A$ are quite different. The columns of the matrix $D$ are called ``atoms'' and are supposed to embody typical patterns observed throughout the columns of $X$, while the columns of $A$ specify the linear combination of columns of $D$ that better approximates the corresponding columns of $X$.

The typical, basic approach to Dictionary Learning is to obtain a local solution by alternate minimization in $D$ and $A$,

\begin{eqnarray}
A\iter{k+1} =& \arg\min_{A} \{ f(D\iter{k}-A) + g(A) \} \\
D\iter{k+1} =& \arg\min_{D} \{ f(D-A\iter{k+1}) + g(A\iter{k+1}) \},
\label{eq:dl}
\end{eqnarray}

where $f(\cdot)$ and $g(\cdot)$ are \emph{fitting} and \emph{regularization} functions respectively. We now briefly discuss two popular Dictionary Learning methods; most of the techniques developed later on (see~\cite{dl-review}) can be seen as modifications of these two.

\subsection{MOD} 
\label{sec:mod}

For the case $f(D,A)=\|DA-X\|_2^2$ and $g(A)=\sum_{i}\|A_i\|_1$ the \emph{Method of Directions} (MOD)~\cite{mod}, is given by
\begin{eqnarray}
A_j\iter{k+1}\!\! &=&\!\! \arg\min_{a \in \reals^p} \{ \|x_j - D\iter{k}a \|_2^2 + \|a\|_1, \} \\
D_r\iter{k+1}\!\! &=&\!\! u_r/\min\{1,\|u_r\|_2\},\\
\!\!&&\!\!u_r = X(A\iter{k+1})_r\transp\left(A\iter{k+1}(A\iter{k+1})\transp\right)^{-1}\!\!\!,\,
\label{eq:mod}
\end{eqnarray}
where $A_j$ and $D_r$ are the $j$-th and $r$-th columns of $A$ and $D$ respectively. The first step corresponds to an $\ell_1$-regularized least squares regression problem on each column of $A$, also known as LASSO~\cite{lasso}.
In the second step, each atom $D_r$ of the dictionary corresponds to a normalized down version of the least squares solution $u_r$. Here too, (although ot ``officially'' part of the algorithm,) it is customary  to apply some sort of regularization so that $AA\transp$ is invertible and well conditioned.

\subsection{K-SVD} 
\label{sec:ksvd}

In this case, $f(\cdot)$ is again the squared $\ell_2$ norm and $g(\cdot)$ corresponds to the $\ell_0$ pseudo-norm, the Hamming weight. 

\paragraph{Coefficients update} The columns of $A$ are computed using a greedy method known as OMP (Orthogonal Matching Pursuit)~\cite{omp}, which   under certain conditions can be shown to provide the actual solution to the $\ell_0$ corresponding penalized least squares problem (see~\cite{tropp07}). A simpler variant of this step uses the (non-orthogonal) Matching Pursuit (MP)~\cite{mp}, which is described next in Algorithm~\ref{alg:mp}, 
%
\begin{algorithm}[ht] 
\caption{\label{alg:mp}Matching Pursuit}
\KwData{vector to encode $x$, dictionary $D$, maximum residual norm $\epsilon$}
\KwResult{Optimal coefficients for $x$, $a$}
Set iteration $k=0$, residual $r\iter{0}=x$, coefficients $a\iter{0}=0$\;
\While{$\norm{r\iter{k}} \geq \epsilon$}{
  $i = \arg\max \left\{ D_i\transp r\iter{k} \right\} $ \;
  $a_i \leftarrow D_i\transp r\iter{k}$ \;
  $r\iter{k+1} \leftarrow r\iter{k} - a_iD_i $\; 
  $k \leftarrow k+1 $ \;
}
\end{algorithm}
%
What MP does at each iteration is to project the residual onto the atom that is most correlated to it, and then remove the projection from the residual. For this to work well, the atoms must be normalized to have $\ell_2$ norm 1.

\paragraph{Dictionary update} The second stage, instead of being a block descent on $D$, performs a rank-one update of each atom $D_r$ and the sub-row formed by the non-zero entries of the row $A^r$, which we call $A^r_{nz}$. Let $R = X - D\iter{k}A\iter{k} + D\iter{k}_r(A\iter{k})^r,$ be the residual matrix obtained by removing the contribution of $D\iter{k}_r$ from the current approximation of $X$. The values of $D_r$ and $A_{nz}^r$ are then replaced by the first pair of left and right eigenvectors of the SVD decomposition of $R$, 
\begin{equation}
D_r = U_1,\quad A_{nz}^r=V_1,\quad U\Sigma V\transp=R.
\label{eq:ksvd3}
\end{equation}
The above procedure is performed sequentially for each atom, from $r=1$ to $r=p$. Note that, since $A$ is updated as well, more passes could be performed in this fashion. K-SVD however performs this only once, thus finishing the alternate minimization iteration.

The K-SVD dictionary step is significantly more costly than that of MOD, but usually requires significantly less iterations to converge to a good result. MOD, is better suited for online dictionary adaptation, as fast approximations of the statistics $AA\transp$ (which can be thought of as the Hessian associated to minimizing $D$)  and $XA\transp$ can be efficiently updated on a sample to sample basis.


\section{Proximus}
\label{sec:proximus}

\falta{Mover notacion aca; ver si quedo bien pegado}

\def\xor{\oplus}


Let us now formalize the notation used for binary operations hereafter. We use $a \land b$ to denote the logical AND operation between binary operators $a$ and $b$; $a \lor b$ denotes logical OR, and $a \xor b$ the eXclusive OR (XOR); logical negation (NEG) is denoted by the unary operator, e.g., $\neg a$.

% son BiMax, ISA, SAMBA and BND, de los Proximus guys

\def\indicator{\mathbf{1}}
Let the indicator function $\indicator(\cdot)$ be defined so that $\indicator(cond)=1$ if $cond$ is true, and $0$ otherwise. Also, let $a \oplus b$ denote modulo-2 addition, also known as eXclusive OR (XOR). The basic Proximus algorithm~\cite{proximus} provides an approximation to the rank-one matrix $uv\transp$ that is closest to the given matrix $X$ in Hamming distance,
\def\bool{\mathrm{bool}}
%
\begin{algorithm} 
\caption{\label{alg:proximus} Proximus}
\KwData{matrix $X \in \{0,1\}^{m{\times}n}$, $u\iter{0} \in \{0,1\}^m$, $v\iter{0} \in \{0,1\}^n$ }
\KwResult{Vectors $u$, $v$ so that $X \approx uv\transp$}
Set iteration $k=0$\;
\Repeat {$u\iter{k}(v\iter{k})\transp = u\iter{k-1}(v\iter{k-1})\transp$} {
  $u\iter{k+1}_i\!\! \leftarrow \indicator \left( A^i v\iter{k} > \|v\iter{k}\|_0/2 \right),\;i=1,\ldots,m$ \;
  $v\iter{k+1}_j\!\! \leftarrow \indicator\left(A_j\transp{u\iter{k+1}}\!>\!\|u\iter{k+1}\|_0/2 \right),j=1,\ldots,n$ \;
  $k \leftarrow k+1 $ \;
}
\end{algorithm}
%
\begin{proposition}
The output $(u,v)$ of the Proximus Algorithm~\ref{alg:proximus} is a local optimum of the problem $\min \|X - uv\transp\|_0$. 
\end{proposition}
\begin{proof}
 Given $v\iter{k}$, it is easy to check that the update $u\iter{k+1}$ in Algorithm~\ref{alg:proximus} is the value of $u$ that \emph{globally} minimizes $\|X \oplus uv\iter{k+1} \|$ (if $ s\iter{k}_i = w\iter{k}/2$, both $0$ and $1$ are equally optima; in such case, we default to $0$). The same happens with the update $v\iter{k+1}$ given $u\iter{k+1}$. Therefore, $h(E\iter{k})=h(X \oplus u\iter{k}(v\iter{k})\transp)$ cannot increase with $k$. As $h(E\iter{k}) \geq 0$ is bounded, non-increasing, and the iterates can take on a finite number of values, the sequence $h(E\iter{k})$ must converge  after a finite number of steps. Let $(u,v)$ be the arguments at which the stopping condition is satisfied. By definition of the algorithm, no change in $u$ or $v$ decreases the objective. This guarantees that $(u,v)$ is a local minimum in a Hamming ball of radius at least $1$.\footnote{notice that we cannot guarantee that a simultaneous change in a single coordinate of $u$ and a single coordinate of $v$ will not decreasae the cost function!.} 
\end{proof}

\subsection{Initialization}
\label{sec:proximus:init}

Initialization is always a serious issue in non-convex problems, particularly in dictionary learning methods. The binary case is no exception. The authors of Proximus propose a number of initialization methods, namely: 

\paragraph{partition} The authors propose to choose one \emph{separator} column (they do not specify how) and then use that column, along with the Hamming centroid of the rows for which the corresponding value in the separator column is $1$.
\paragraph{greedy graph growing} Let $P$ be a subset of rows whose only initial element is a randomly select row. Then, recursively add all rows which share a non-zero location with \emph{any} row currently in $P$ until no additional rows can be added. The initial pattern vector $v$ is the Hamming centroid of the rows in $P$.
\paragraph{neighbor} Similar to greedy graph growing, but less greedy, this one chooses a pivot row at random and initializes the pattern vector to the Hamming centroid of all the rows that share at least one non-zero location with the pivot row.

The above initialization heuristics are not claimed to be good in any sense; they just embody some principles and practices usually seen in Data Mining applications. A careful analysis of each method is beyond this paper. However, it is important to shed some intuition as to how these methods would perform in the $n \gg m$ setting found in dictionary learning problems.
Also, it is important to consider that these initialization methods are designed for a rank-one decomposition and might be unsuitable or require modifications to work for rank-k factorizations. We will discuss these issues in the context of the proposed methods, which are described in the next section.

\subsection{Binary factorizations beyond rank one}
\falta{Esto debería estar antes, fuera de Proximus}
 
Now we return to our previous discussion. For rank one, it is easy to interpret the product $\hat{X}=uv\transp$; the element $\hat{x}_{ij}$ will be $1$ only $u_i$ and $v_j$ are $1$, that is, if we denote the AND operation by $\land$,  $\hat{x}_{ij} = u_i \odot v_j$. Furthermore, this product coincides with the \emph{standard} product for binary operators. If a rank-k, $k>1$ model is sought, even with binary factors, there are several ways in which the ranks can be combined to produce a result. One is to use modulo-2 arithmetic, that is, $\hat{X} = u_1v_1\transp \oplus u_2v_2\transp \oplus \ldots \oplus u_kv_k\transp$. Other is to use Boolean algebra, where addition is mapped to the OR operator, and we have $\hat{X} = u_1v_1\transp \lor u_2v_2\transp \lor \ldots \lor u_kv_k\transp$. Finally, we can use standard addition and then let the result be defined by some threshold, for example,
$\hat{X} = \indicator\left(\sum_{r=1}^k u_kv_k\transp \geq k/2 \right)$.
Which one is better will ultimately depend on the task and the prior information available about $X$. We choose to work with modulo-2 algebra as it enables the set $\{0,1\}$ to be treated as a numeric Field, featuring neutral sum, neutral product, inverse and negative elements. An immediate practical consequence is the possibility to \emph{substract}, that is $-1=1$ and $1 - 1 = 1 + 1 = 1 \oplus 1 = 0$, which in turn enables higher ranks to correct systematic approximation errors due to lower rank models. (Note that this is not possible with, for example, Boolean algebra, as $1+1=1\lor 1$ and $1-1$ is undefined, so a $1$ cannot be turned back to $0$.)

\section{Binary Dictionary Learning}
\label{sec:bdl}

Given a particular model order (determined by the dictionary size) $p$, our proposed method consists of a traditional alternate descent dictionary learning approach much like MOD and K-SVD. On top of it, we use the Minimum Description Length (MDL) criterion for evaluating the parsimony of the best learned model of size $p$, together with  three different algorithms for performing the selection of the best model among a range of model sizes $p=p_{\min},\ldots,p_{\max}$. We leave the description of the top-level model selection algorithms to Section~\ref{sec:model-selection}.

For the dictionary learning algorithm we provide a common coefficients update step, the Binary Matching Pursuit (BMP) algorithm, and  two choices for the dictionary update step: MOB (a binarized version of MOD) and K-PROX (combining ideas of K-SVD and Proximus)  We now provide details on each of these algorithms

\paragraph{Coefficients update via Binary Matching Pursuit (BMP)}

In essence,  BMP as a binarized variant of MP. However, there are some subtleties which need to be addressed. We remind the reader that $a \oplus b$ indicates modulo-2 addition. We also define $A \odot a$ to be the matrix-vector modulo-2 product, that is, for a matrix $A \in \{0,1\}^{m{\times}n}$, $A \odot a = a_0 \land A_0 \oplus a_1 \land A_1 \ldots \oplus a_n \land A_n$, where $a_i \land A_i$ is $A_i$ if $a_i = 1$ and the all-zeros vector if $a_i = 0$.  Regular operations such as integer/real/complex summation or matrix-vector product are written as usual, $a+b$, $Ax$, etc.

For a given sample $x$, we begin ($t=0$) with an initial coefficients vector $a\iter{0}=a_0$, a residual $r\iter{0}_j=x \oplus D \odot a\iter{0}$ and an initial vector $g\iter{0}=D\transp x$ which records the \emph{Euclidean} (traditional)  correlation between the columns of $D$ and $x$; this detail will be clarified later. Then, at each iteration $t$ we \emph{toggle} the coefficient $a_j(k)$ corresponding to the atom that is most correlated to $r\iter{t}$, $D_{k}$, $D_k$ is accordingly added to $r\iter{t}$. The pseudocode is given in Algorithm~\ref{alg:bmp}. 
%
\begin{algorithm}[ht]
\KwData{sample to encode $x$, dictionary $D$, initial coeficients $a_0$, maximum steps $t_{\max}$, maximum residual weight $w_{\max}$, residual exponent $\beta$}
\KwResult{Optimum coefficients for $x$, $a$}
Set iteration $t=0$, coefficients $a\iter{0}=a_0$ ,residual $r\iter{0}=x \oplus Da\iter{0}$\;
Set modulo-2 Gramm matrix $G = D\transp \odot D$\tcc*{(a)}
Set residual correlation $g\iter{0}=D\transp{r\iter{0}}$\tcc*{(b)}
\While{$h(r\iter{t}) \geq w_{\max}$ \textbf{and} $t < t_{\max}$ }{
  $k \leftarrow \arg \max_l \{\;|g\iter{t}_l|\;/\;\|D_l\|_2^\beta\;\} $ \tcc*{(c)}
  \If{$g\iter{t}_{k} = 0$} { \textbf{finish} }  
  $r\iter{t+1} \leftarrow r\iter{t} \oplus  D_{k} $\; 
  \If { $h(r\iter{t+1}) \geq h(r\iter{t})$\tcc*{(d)} } 
  {
	  $r\iter{t+1} \leftarrow r\iter{t} \oplus  D_{k} $\; 
	  \textbf{finish} \;
  }
  \eIf {$a\iter{t+1}_{k} = 1$} {
    $a\iter{t+1}_{k} \leftarrow 0$ \;
    $g\iter{t+1} \leftarrow g\iter{t} - G_{k}$\tcc*{(e)}
    } {
    $a\iter{t+1}_{k} = 1$ \; 
    $g\iter{t+1} \leftarrow g\iter{t} + G_{k}$ \tcc*{(e)}
    }

} 
\caption{\label{alg:bmp} Binary Matching Pursuit}
\end{algorithm}

Some steps of the algorithm, marked as $(a)$, $(b)$, $(c)$, $(d)$ and $(e)$ (appearing twice) are not obvious from the overall description of the algorithm and need to be clarified. 

In $(a)$ we see the creation of an auxiliary matrix, the \emph{modulo-2 Gramm matrix} of $D$. In the traditional OMP and MP algorithms, this matrix is used in the same way throughout the algorithm, but is computed using traditional arithmetic. The idea is that the correlations vector can be efficiently updated when the new coefficients vector $a\iter{t+1}$ differs from $a\iter{t}$ by a value of $\Delta$ at the $k$-th coordinate: $g\iter{t+1} = D\transp (r\iter{t} - \Delta D_k) = g\iter{t} - \Delta D\transp D_k = g\iter{t} - \Delta\transp G_k$; this requires only $2p$ operations instead of the $\approx 2pm$ required if the correlations vector was computed all over again.

Despite the fact that we are working with binary vectors with binary operations, the euclidean correlation between the dictionary and the residual is still a good proxy for choosing an atom that will serve to reduce the weight of the residual. Note however that this will work correctly if the atoms are normalized, which they are not. This is why 
the maximum in $(c)$ is taken with respect to the ration $g\iter{t}_l / \}D_l\|_2^\beta$. If we wanted to truly normalize the atoms, we should use the Euclidean norm of $D_l$, that is, $\beta=1$. However, we leave this as a parameter, as it is useful to increase $\beta$ above $1$, which favors atoms with lower weights. We have found in our experiments that $\beta=2$
(in which case $\|D_l\|_2^2=h(D_l)$) gives better results than $\beta=1$.

The step marked with $(d)$ adds monotonicity to the method: if the weight of the residual increases (which might happen), the update is rolled back and the algorithm finishes.

Finally, and most interestingly, despite the correlation vector being initially computed using a \emph{traditional} matrix-vector product, its update can be written as the traditional update used in OMP/MP, only that the Gramm matrix is actually computed using the modulo-2. The first $(e)$ corresponds to the case when $\Delta=1$, that is, when $a_k$ is switched from $0$ to $1$. The second $(e)$ corresponds to $\Delta=-1$, when $a_k$ is switched off. (This curious result is easily verified by writing down the corresponding arithmetic.) 

%
%\subsection{Coefficients update/Binary Weight Pursuit (BWP)}
%
%\begin{algorithm}[ht]
%\KwData{sample to encode $x$, dictionary $D$, initial coeficients $a$, maximum steps $t_{\max}$, maximum residual weight $w_{\max}$}
%\KwResult{Optimum coefficients for $x$, $a$}
%Set iteration $t=0$, residual $r\iter{0}=x$, coefficients $a\iter{0}=a$\;
%Set residual correlation $g\iter{0}=D\transp{r\iter{0}}$ \;
%\While{$h(r\iter{k}) \geq w_{\max}$ \textbf{and} $t < t_{\max}$ }{
%  $h_{\min} \leftarrow \min h(D_i \oplus r\iter{k}) $ \;
%  \If{$h_{\min} > h(r\iter{k})$} { break }  
%  $i \leftarrow \arg\min h(D_i,r\iter{k}) $ \;
%  $a_i  \leftarrow  {a}_i  \oplus 1 $ \;
%  $r\iter{k+1} \leftarrow r\iter{k} \oplus  D_i $\; 
%  
%}
%\label{alg:bmp}
%\end{algorithm}
\subsection{MOB: Method Of Binary directions}
\label{sec:bdl:mod}
 
Here we want to update each atom so that the Hamming weight of the total residual matrix $E = X \oplus D \odot A$ is minimum (we recall that minus and plus are the same thing in modulo-2 arithmetic). Say we want to update the $r$-th atom at iteration $k$. Clearly, the affected columns will only be those for which the  coefficients in $A$  corresponding to that atom are non-zero, that is $\{j : A_{rj} \neq 0 \}$; let us call this set $J_r$ and $n_r$ its size. What we want is the update $\Delta$ that, when added to $D_r$, minimizes the weight of the residual $E$ in those colums affected by $D_r$,
 \begin{eqnarray}
 \Delta  =& \arg\min_d \sum_{j \in J_r}  h(E\iter{k}_j \oplus d) \\
 =& \arg\min_d \sum_{j \in J_r} (\sum_i E\iter{k}_j + n_r d_i).
\label{eq:mob1}
 \end{eqnarray}

The above objective is trivially separable in the elements of $d$,
 \begin{eqnarray}
 \Delta_i  =& \arg\min_{u \in \{0,1\}} \sum_j E_{ij}\iter{k} \oplus d_i\\
 =& \arg\min_{u \in \{0,1\}} \sum_{j\in J_r} E_{ij}\iter{k} + n_r u.
\label{eq:mob2}
 \end{eqnarray}
In other words, the $i$-th element of $D_r$ should be $0$ if most of the elements in the rows of $E$ affected by it are $0$, and $1$ otherwise. 
This is clearly an optimum solution (note that there could be many optima  when $n_r$ is even, in which case we simply choose one value; in our case we use $0$).

\subsection{K-PROX: Dictionary Update via Proximus}
\label{sec:bdl:k-prox}

In this case, following the K-SVD concept, we want to obtain the best rank-one approximation to the residual $E$ obtained after removing the contribution of $D_r$. Let $J_r = \{j: A_{rj} \neq 0 \}$ and $E_{J_r}$, $X_{J_r}$ and $A_{J_r}$ the submatrices formed by the columns of $X$ and $A$ indexed by $J_r$ and $R_{J_r}=D{\odot}A_{J_r} \oplus E_{J_r}$. We then have the following update:
\begin{equation}
(D_r\iter{k+1},A_{J_r}\iter{k+1}) = \arg\min_{u,v} h\left( R_{J_r}\iter{k} \oplus uv\transp \right).
\label{eq:bsvd1}
\end{equation}
As the name K-PROX implies, we seek an approximation to the NP-hard problem \refeq{eq:bsvd1} using the Proximus algorithm defined in Section~\ref{sec:proximus}.

\subsection{Initialization}

As mentioned before, initialization is of paramount importance to the success of any non-convex matrix factorization method. At the same time, there is no provably optimum way of doing so, otherwise we would be contravening the NP-hard nature of the factorization problem itself. We are left with heuristics based on intuition and prior information, if any. Ultimately, it is an art. Below we describe the three methods we have tested with our algorithms, two of which are vaguely inspired by the ones proposed in Proximus, which are defined only for rank-$1$ factorizations.

In any case, these are not to be taken as particularly good methods; they are examples which in some cases, as we will see, can work well. We advise the interested reader to investigate whether these are adequate to the data at hand in each case, eventually defining new strategies which can take into account prior information particular to the problem at hand. 

\paragraph{(Pseudo) Partition}: This method works for dictionaries of size $p \leq m$. Let $I=\{i_1,i_2,\ldots,i_p\}$ be the indexes of the $p$ rows of $X$ which have the largest weight, that is, $h(X_i) \geq h(X_{i'}),\forall\,i\in{I},\,i' \notin{I}$. Then the $k$-th atom is initialized as the Hamming average of all the \emph{columns} of $X$ which have a $1$ in their $i_k$-th row. If we define $J_k=\{j:1 \leq j \leq n, X_{ij_k}=1 \}$, then $$D_k = \frac{1}{|J_k|}\sum_{j \in J_k} X_j.$$ 
The idea behind this initialization is to ensure that the initial atoms are correlated with those dimensions which are more commonly active in the dataset; this is particularly useful in very sparse matrices, but for the same reason can fail for dense matrices as all the atoms may end up being mostly $1$ and very similar among themselves.
 
\paragraph{Neighbor}: Here we draw $p$ samples at random from $X$, then each atom $D_k$ is initialized as the Hamming average of all samples in $X$ which have non-zero correlation with $X$. This is essentially a rank-$k$ extension of the method described in \cite{proximus} under the same name. It is however not a good method for dense, fat matrices (see Figure~\ref{fig:neighbor-fail} on page~\pageref{fig:neighbor-fail}), especially if some row of $E$ is largely composed of $1$s, in which case all or most atoms may end up being identical, dense patterns.

\paragraph{Random} This is a fallback, universal alternative for initializing a dictionary when almost no prior information can be exploited to devise a better alternative. The samples are initialized using pseudo-random Bernoulli samples of probability $\theta$, a parameter which can be chosen to reflect, for example, the density of $1$s in the dataset.  The value by default is $1/2$, which is a good idea even for very sparse matrices, as $\theta \ll 1$ combined with a large dimension $m$ may result in atoms which are orthogonal to most or even all data samples if the data is sparse.


%\subsection{Learning variants}
%
%I implemented several variants of the above algorithms, most of them just change the order in which some updates are made, or switch the roles of dictionary and coefficients alternatively, so that the method can be used on a broader class of data.
% 
%
%\begin{enumerate}
%\item Traditional: traditional alternate descent until local convergence
%\item Role-switching I: at each iteration, the role of A and D are switched
%\item Role-switched learning II: after local convergence (the same as in the first case), the role of A and D are switched and the traditional model is applied again
%\item Role switched learning III: like type I but only the dictionary update step is applied (for use with Proximus)
%\end{enumerate}

\section{Model selection}
\label{sec:model-selection}

The following two methods are actually meant to choose the best from an ensemble of candidate dictionaries learned using one of the previous four methods (referred to as ``inner learning method'' in the implementation). MDL stands for Minimum Description Length, and is a criterion for model selection (that is, choosing the best model out of a set of competing models for describing some particular data) where the criterion used is, essentially, ``which of the models produces a more succint description of the data''; in other words, which model compresses the most.
 
\falta{describir claramente como se aplica MDL aca}
 
\begin{enumerate}
\item MDL/forward selection:  Here begin with an initial dictionary size $K_0$, train a dictionary, add a few more atoms to obtain $K_1$, and keep doing this until no further improvement is obtained, that is, until the overall codelength does not decrease by adding new atoms to the dictionary.
 
\item MDL/backward selection: We begin with a maximum dictionary size $K_0$, train the dictionary, and then remove the worst atom using some criterion (for example, remove one atom at a time and drop the one  which results in the largest decrease in the overall codelength). If removing an atom does not result in a better overall codelength, the method stops. 

\item MDL/full search: in both forward and backward selection, all or a part of the dictionary used in one iteration is kept and re-adapted after adding or removing atoms from it. In this case, for a range of values of $K$, a whole new dictionary is trained, and the best one is chosen among them. This method is much slower than the previous two ones (the fastest is forward selection), but tends to provide better results; This tradeoff is open for research.

\end{enumerate}


%\section{Applications}
%
%Here we describe some classical applications from the Dictionary Learning literature, with some references to some of the best results in each case.


\section{Denoising}
\label{sec:denoising}

Denoising is a special case of the more general \emph{signal restoration} problem (other examples are zooming, deblurring, or inpainting -- a.k.a. filling erased regions). Here we assume that we want to recover a signal sample $x \in \reals^{m}$ from a noisy observation $z \in \reals^{m}$, both related by
\[
z = x + n,
\]
where $n \in \reals^m$ is a vector of i.i.d. samples from some known distribution, e.g. Gaussian or Poisson. For instance, in the context of Image Processing, $z$ and $x$ are usually vectorized versions of square $\sqrt{m}\times\sqrt{m}$ patches of an image, and what one desires to recover is the whole image $I$ from its corresponding noisy version $J$.

The method used for denoising an image in the Dictionary Learning framework usually proceeds as follows:
\begin{itemize}
\item An initial dictionary $D_0 \in \reals^{m{\times}p}$ is available; this dictionary is learned to efficiently represent a \emph{large} set of patches taken from some public dataset of natural images, usually comprising thousands of images.
\item The image to be denoised, $J \in \reals^{M{\times}N}$, is decomposed into $n=(M-\sqrt{m}+1){\times}(N-\sqrt{m}+1)$ overlapping patches $(z_j : j = 1,\ldots,n)$ that we will arrange for convenience as columns in a matrix $Z \in \reals^{m{\times}n}$
\item The samples $Z$ are used to further adapt the initial dictionary and the accompanying sparse coefficients $A \in \reals^{p{\times}n}$; the sparse coding variant during the coefficient update step of this dictionary learning process is the \emph{denoising} formulation, given by
\[
a\iter{t+1}_j = \arg\min_a \|a\|_r \st \|z_j - D\iter{t}a \|_2 \leq C\sigma, 
\]
where $\sigma$ is the variance of the noise $C$ is a positive constant usually close to $1$ and $0 \leq r \leq 1$ is a sparsity-inducing (pseudo)-norm. The dictionary update is the same as described before in \refeq{sec:mod}
\item Upon convergence, the \emph{clean image patches $x_j$} are estimated from the solution $(D\opt,A\opt)$ as $$x_j = D\opt a\opt_j.$$
\item Finally, the estimated clean image is obtained by stitching back the estimated clean patches $x_j$ into their respective locations, averaging pixels where intersections between patches occur.
\end{itemize}

%\subsection{Inpainting -- missing data}
%
%Here the task is to fill in missing samples. As we only have binary matrices around, we need an auxiliary mask matrix $H \in \reals^{m{\times}n}$ which will tell us which samples are to be considered missing (value $0$) in the data matrix $X$, regardless of their value in $X$. The known values in $X$ are indicated with a $1$ in the corresponding place in $H$. The task is to fill in the missing values.
%
%If we have a dictionary $D$ trained to samples similar to $X$, the idea is the following:
%
%\begin{itemize}
%\item For each sample $x$ with missing samples specified in a corresponding mask sample $h$,
%\item Take the subset of rows from $D$ and $x$, $D_{h}$ and  $x_{h}$ for which the corresponding row in $h$ is $1$.
%\item Encode $x_{h}$ using $D_{h}$ using the usual encoding scheme (that is, add atoms until no further decrease in the error is obtained); we obtain a vector of coefficients $a$
%\item Fill in the subvector of missing values, $x_{\bar h}$ ($\bar h = 1 - h$) as, $$x_{\bar h} = D_{\bar h}a$$
%\end{itemize}

%\subsection{Classification}
%
%Here again we discuss the case of image patches classification, including the special case where the patches are not part of a larger image but pre-cropped and aligned handwritten characters taken from the public datasets MNIST and USPS typically used in character recognition benchmarks.
%
%There are several variants in this case. The one we'll mention here~\cite{ramirez10cvpr}, which seems quite natural, is a generative one where a dictionary $D_c$ is  adapted to efficiently represent samples from each class $c  \ in \mathcal{C}$. A new sample is then classified into class $c'$ if its representation under the dictionary $D_{c'}$ is better in some sense than that obtained using the dictionaries trained for the other classes. The method  can be summarized as follows:
%
%\begin{itemize}
%\item A  dictionary $D_c$ is adapted to a set of samples $X_c$ known to belong to class $c$ using some known method; this is done for each $c \in \mathcal{C}$. 
%\item In order  to classifiy a new sample $x$, it is sparsely encoded using each of the dictionaries $D_c, c \in \mathcal{C}$, and the corresponding optimum coding cost is used as a score $l_c$. Any sparse coding variant could be used in principle, but typical choices here are the ``basis pursuit'' variant,
%\[
%l_c(x) = \min_a \|z_j - D\iter{t}a \|_2 \st  \|a\|_r \leq \tau, 
%\]
%and the Lagrangian variant,
%\[
%l_c(x) = \min_a \|z_j - D\iter{t}a \|_2 + \tau  \|a\|_r. 
%\]
%\item The sample $x$ is then declared to belong to the class $c$ so that  $l_c$ is the smallest. 
%More so than in denoising, the success in this application depends largely  on the (critical) parameters $\tau$,  the $r$-norm  to be used, and the size of the dictionary $p$. These critical issues are treated in \cite{ramirez12tsp}.
%\end{itemize}

\section{Results and discussion}
\label{sec:results}

\subsection{Modeling examples}

\subsection{Denoising}

%FALTA
%
%\subsection{Missing data}
%
%FALTA
%
%\subsection{Classification}
%
%FALTA
%
\section{Conclusion}
\label{sec:conclusion}
%
%FALTA

\bibliographystyle{plain}
\bibliography{IEEEabrv,bmf}

\end{document}
